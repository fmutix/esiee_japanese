\documentclass[11pt]{report}
\usepackage[top=2cm, bottom=2cm, left=2cm, right=2cm]{geometry}
\usepackage[T1]{fontenc}
\usepackage[utf8]{inputenc}
\usepackage[french]{babel}
\usepackage{lmodern}
\usepackage{hyperref}
\usepackage{titlesec, blindtext}
\usepackage{multirow}
\usepackage{listings}
\usepackage{color}
\usepackage{fixltx2e}
\usepackage{graphicx}
\usepackage{amsmath}
\usepackage{CJKutf8}
\newenvironment{Japanese}{%  
\CJKfamily{min}%  
\CJKtilde  
\CJKnospace}{} 

\definecolor{dkgreen}{rgb}{0,0.6,0}
\definecolor{gray}{rgb}{0.5,0.5,0.5}
\definecolor{mauve}{rgb}{0.58,0,0.82}
\definecolor{blue}{rgb}{0,0,0.7}
\renewcommand{\arraystretch}{1.2}
\newcommand{\hsp}{\hspace{20pt}}
\titleformat{\chapter}[hang]{\Huge\bfseries}{\thechapter\hsp{|}\hsp}{0pt}{\Huge\bfseries}
\hypersetup{
	colorlinks=true,       	% false: boxed links; true: colored links
	linkcolor=black,          	% color of internal links
	urlcolor=blue           	% color of external links
}

\title{Cours de japonais}
\author{
	Frédéric Nguyen \\ Club*Nix
}

\begin{document}
\maketitle
\tableofcontents

\chapter{Ichidan}

\begin{CJK}{UTF8}{}  
\begin{Japanese}
	\begin{center}
		\begin{tabular}{|c|c|c|}
				\hline
				\textbf{Français} & \textbf{Rõmaji} & \textbf{Hiragana -masu} \\
				\hline
				se coucher & nemasu & ねます \\	
				\hline 
				prendre une douche & abimasu & あびます \\
				\hline
				prendre une photo & torimasu & とります \\
				\hline
				se lever & okimasu & おきます \\
				\hline
				manger & tabemasu & たべます \\
				\hline
				rencontrer & aimasu & あいます \\
				\hline
			\end{tabular}
	\end{center}
\end{Japanese}  
\end{CJK}

\chapter{Godan}

\begin{CJK}{UTF8}{}  
\begin{Japanese}
	\begin{center}
		\begin{tabular}{|c|c|c|}
				\hline
				\textbf{Français} & \textbf{Rõmaji} & \textbf{Hiragana -masu} \\
				\hline
				acheter & kaimasu & かいます \\
				\hline
				aller & ikimasu & いきます \\
				\hline
				boire & nomimasu & のみます \\
				\hline
				se brosser les dents & migakimasu & みがきます \\
				\hline
				écouter & kikimasu & ききます \\
				\hline
				écrire & kakimasu & かきます \\
				\hline
				être/se situer (objet) & arimasu & あります \\
				\hline
				se déshabiller & nugimasu & ぬぎます \\
				\hline
				lire & yomimasu & よみます \\
				\hline
				parler & hanashimasu & はなします \\
				\hline
				regarder & mimasu & みます \\
				\hline
				rentrer & kaerimasu & かえります \\
				\hline
				téléphoner & kakemasu & かけます \\
				\hline
				terminer & owarimasu & おわります \\
				\hline
			\end{tabular}
	\end{center}
\end{Japanese}  
\end{CJK}

\chapter{Irrégulier}

\begin{CJK}{UTF8}{}  
\begin{Japanese}
	\begin{center}
		\begin{tabular}{|c|c|c|}
				\hline
				\textbf{Français} & \textbf{Rõmaji} & \textbf{Hiragana -masu} \\
				\hline
				faire & shimasu & します \\
				\hline
			\end{tabular}
	\end{center}
\end{Japanese}  
\end{CJK}

\end{document}