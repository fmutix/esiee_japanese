\documentclass[11pt]{report}
\usepackage[top=2cm, bottom=2cm, left=2cm, right=2cm]{geometry}
\usepackage[T1]{fontenc}
\usepackage[utf8]{inputenc}
\usepackage[french]{babel}
\usepackage{lmodern}
\usepackage{hyperref}
\usepackage{titlesec, blindtext}
\usepackage{multirow}
\usepackage{listings}
\usepackage{color}
\usepackage{fixltx2e}
\usepackage{graphicx}
\usepackage{amsmath}
\usepackage{CJKutf8}
\newenvironment{Japanese}{%  
\CJKfamily{min}%  
\CJKtilde  
\CJKnospace}{} 

\definecolor{dkgreen}{rgb}{0,0.6,0}
\definecolor{gray}{rgb}{0.5,0.5,0.5}
\definecolor{mauve}{rgb}{0.58,0,0.82}
\definecolor{blue}{rgb}{0,0,0.7}
\renewcommand{\arraystretch}{1.2}
\newcommand{\hsp}{\hspace{20pt}}
\titleformat{\chapter}[hang]{\Huge\bfseries}{\thechapter\hsp{|}\hsp}{0pt}{\Huge\bfseries}
\hypersetup{
	colorlinks=true,       	% false: boxed links; true: colored links
	linkcolor=black,          	% color of internal links
	urlcolor=blue           	% color of external links
}

\title{Cours de japonais}
\author{
	Frédéric Nguyen \\ Club*Nix
}

\begin{document}
\maketitle
\tableofcontents

\chapter{Ichidan}

\begin{CJK}{UTF8}{}  
\begin{Japanese}
	\begin{center}
		\begin{tabular}{|c|c|c|}
				\hline
				\textbf{Français} & \textbf{Rõmaji} & \textbf{Hiragana -masu} \\
				\hline
				se \textbf{coucher} & nemasu & ねます \\	
				\hline 
				\textbf{descendre} de & orimasu & おります \\
				\hline
				\textbf{enseigner/indiquer} & oshiemasu & おしえます \\
				\hline
				\textbf{fermer} & shimemasu & しめます \\
				\hline
				se \textbf{lever} & okimasu & おきます \\
				\hline
				\textbf{manger} & tabemasu & たべます \\
				\hline
				\textbf{montrer} & misemasu & みせます \\
				\hline
				\textbf{ouvrir} & akemasu & あけます \\
				\hline
				\textbf{prendre une douche} & abimasu & あびます \\
				\hline
				\textbf{prendre une photo} & torimasu & とります \\
				\hline
				\textbf{rencontrer} & aimasu & あいます \\
				\hline
			\end{tabular}
	\end{center}
\end{Japanese}  
\end{CJK}

\chapter{Godan}

\begin{CJK}{UTF8}{}  
\begin{Japanese}
	\begin{center}
		\begin{tabular}{|c|c|c|}
				\hline
				\textbf{Français} & \textbf{Rõmaji} & \textbf{Hiragana -masu} \\
				\hline
				\textbf{acheter} & kaimasu & かいます \\
				\hline
				\textbf{aller} & ikimasu & いきます \\
				\hline
				\textbf{appeler} (voix) & yobimasu & よびます \\
				\hline
				\textbf{apprendre} & naraimasu & ならいます \\
				\hline
				\textbf{boire} & nomimasu & のみます \\
				\hline
				se \textbf{brosser les dents} & migakimasu & みがきます \\
				\hline
				\textbf{chanter} & utaimasu & うたいます \\
				\hline
				\textbf{créer} & tsukurimasu & つくります \\
				\hline
				\textbf{écouter} & kikimasu & ききます \\
				\hline
				\textbf{écrire} & kakimasu & かきます \\
				\hline
				\textbf{éteindre} & keshimasu & けします \\
				\hline
				\textbf{être/se situer} (objet) & arimasu & あります \\
				\hline
				se \textbf{déshabiller} & nugimasu & ぬぎます \\
				\hline
				se \textbf{laver} & araimasu & あらいます \\
				\hline
				\textbf{lire} & yomimasu & よみます \\
				\hline
				\textbf{monter} dans & norimasu & のります \\
				\hline
				\textbf{parler} & hanashimasu & はなします \\
				\hline
				\textbf{regarder} & mimasu & みます \\
				\hline
				\textbf{rentrer} & kaerimasu & かえります \\
				\hline
				\textbf{sortir} & demasu & でる \\
				\hline
				\textbf{téléphoner} & kakemasu & かけます \\
				\hline
				\textbf{terminer} & owarimasu & おわります \\
				\hline
			\end{tabular}
	\end{center}
\end{Japanese}  
\end{CJK}

\chapter{Irrégulier}

\begin{CJK}{UTF8}{}  
\begin{Japanese}
	\begin{center}
		\begin{tabular}{|c|c|c|}
				\hline
				\textbf{Français} & \textbf{Rõmaji} & \textbf{Hiragana -masu} \\
				\hline
				faire & shimasu & します \\
				\hline
			\end{tabular}
	\end{center}
\end{Japanese}  
\end{CJK}

\end{document}