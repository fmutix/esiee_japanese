\documentclass[11pt]{report}
\usepackage[top=2cm, bottom=2cm, left=2cm, right=2cm]{geometry}
\usepackage[T1]{fontenc}
\usepackage[utf8]{inputenc}
\usepackage[french]{babel}
\usepackage{lmodern}
\usepackage{hyperref}
\usepackage{titlesec, blindtext}
\usepackage{multirow}
\usepackage{listings}
\usepackage{color}
\usepackage{fixltx2e}
\usepackage{graphicx}
\usepackage{amsmath}
\usepackage{CJKutf8}
\newenvironment{Japanese}{%  
\CJKfamily{min}%  
\CJKtilde  
\CJKnospace}{} 

\definecolor{dkgreen}{rgb}{0,0.6,0}
\definecolor{gray}{rgb}{0.5,0.5,0.5}
\definecolor{mauve}{rgb}{0.58,0,0.82}
\definecolor{blue}{rgb}{0,0,0.7}
\renewcommand{\arraystretch}{1.2}
\newcommand{\hsp}{\hspace{20pt}}
\titleformat{\chapter}[hang]{\Huge\bfseries}{\thechapter\hsp{|}\hsp}{0pt}{\Huge\bfseries}
\hypersetup{
	colorlinks=true,       	% false: boxed links; true: colored links
	linkcolor=black,          	% color of internal links
	urlcolor=blue           	% color of external links
}

\usepackage{multicol}
\setlength\columnsep{-13.1em}
%\setlength\columnseprule{.4pt}
\title{Japonais - Adjectifs rencontrés}
\author{
	Frédéric Nguyen \\ Club*Nix
}

\begin{document}
\maketitle
\tableofcontents

\chapter{Adjectifs en i}

\begin{CJK}{UTF8}{}  
\begin{Japanese}
	\begin{center}
		\begin{multicols}{2}
			\begin{tabular}{|c|c|c|}
				\hline
				\textbf{Français} & \textbf{Rõmaji} & \textbf{Hiragana} \\
				\hline
				\textbf{froid} & samui & さむい \\
				\hline
			\end{tabular}
			\begin{tabular}{|c|c|c|}
				\hline
				\textbf{Hiragana} & \textbf{Rõmaji} & \textbf{Français} \\
				\hline
				あつい & atsui & \textbf{chaud}\\
				\hline 		
			\end{tabular}
		\end{multicols}
	\end{center}
\end{Japanese}  
\end{CJK}

\chapter{Adjectifs en na}

\begin{CJK}{UTF8}{}  
\begin{Japanese}
	\begin{center}
		\begin{tabular}{|c|c|c|}

		\end{tabular}
	\end{center}
\end{Japanese}  
\end{CJK}

\end{document}