\documentclass[11pt]{report}
\usepackage[top=2cm, bottom=2cm, left=2cm, right=2cm]{geometry}
\usepackage[T1]{fontenc}
\usepackage[utf8]{inputenc}
\usepackage[french]{babel}
\usepackage{lmodern}
\usepackage{hyperref}
\usepackage{titlesec, blindtext}
\usepackage{multirow}
\usepackage{listings}
\usepackage[usenames,dvipsnames]{color}
\usepackage{fixltx2e}
\usepackage{graphicx}
\usepackage{amsmath}
\usepackage{CJKutf8}
\newenvironment{Japanese}{%  
\CJKfamily{min}%  
\CJKtilde  
\CJKnospace}{} 

\definecolor{dkgreen}{rgb}{0,0.6,0}
\definecolor{gray}{rgb}{0.5,0.5,0.5}
\definecolor{mauve}{rgb}{0.58,0,0.82}
\definecolor{blue}{rgb}{0,0,0.7}
\renewcommand{\arraystretch}{1.2}
\newcommand{\hsp}{\hspace{20pt}}
\titleformat{\chapter}[hang]{\Huge\bfseries}{\thechapter\hsp{|}\hsp}{0pt}{\Huge\bfseries}
\hypersetup{
	colorlinks=true,       	% false: boxed links; true: colored links
	linkcolor=black,          	% color of internal links
	urlcolor=blue           	% color of external links
}

\usepackage{multicol}
\setlength\columnsep{-9.4em}
%\setlength\columnseprule{.4pt}
\title{Japonais - Adjectifs rencontrés}
\author{
	Club*Nix
}

\begin{document}
\maketitle
\tableofcontents

\chapter{Adjectifs en i}

\begin{CJK}{UTF8}{}  
\begin{Japanese}
	\begin{center}
		\begin{multicols}{2}
			\begin{tabular}{|c|c|c|}
				\hline
				\textbf{Français} & \textbf{Rõmaji} & \textbf{Hiragana} \\
				\hline
				\textbf{chaud} & atsui & あつい \\%0
				\hline
				\textbf{chaud} & atsui & あつい \\%1
				\hline
				\textbf{haut} & takai & たかい \\%2
				\hline
				\textbf{cher} & takai & たかい \\%3
				\hline
				\textbf{grand} & ookii & おおきい \\%3
				\hline
				\textbf{long} & nagai & ながい \\%5
				\hline
				\textbf{difficile} & muzukashii & むずかしい \\%6
				\hline
				\textbf{neuf} & atarashii & あたらしい \\%7
				\hline
				\textbf{rapide/t\^ot} & hayai & はやい \\%8 %ma touche pour l'accent circonflexe veux pas fonctinner : (
				\hline
				\textbf{interessant} & omoshiroi & おもしろい \\%9
				\hline
				\textbf{délicieux} & oishii & おいしい \\%10
				\hline
				\textbf{bon} & ii & いい \\%11
				\hline
				\textbf{lourd} & omoi & おもい \\%12
				\hline
				\textbf{content} & ureshii & うれしい \\%13
				\hline
				\textbf{lumineux} & akarui & あかるい \\%14
				\hline
%				\textbf{} &  &  \\%15
%				\hline
%				\textbf{} &  &  \\%16
%				\hline
%				\textbf{} &  &  \\%17
%				\hline
%				\textbf{} &  &  \\%18
%				\hline
%				\textbf{} &  &  \\%19
%				\hline
%				\textbf{} &  &  \\%20
%				\hline

			\end{tabular}
			\begin{tabular}{|c|c|c|}
				\hline
				\textbf{Hiragana} & \textbf{Rõmaji} & \textbf{Français} \\
				\hline
				さむい & samui & \textbf{froid} \\%0
				\hline
				つめたい & tsumetai & \textbf{frais} \\%1
				\hline
				ひくい & hikui & \textbf{bas} \\%2
				\hline
				やすい & yasui & \textbf{pas cher} \\%3
				\hline
				ちいさい & chiisai & \textbf{petit} \\%4
				\hline
				みじかい & mijikai & \textbf{court} \\%5
				\hline
				やさしい & yasashii & \textbf{facile} \\%6
				\hline
				ふるい & furui & \textbf{ancien} \\%7
				\hline
				おそい & osoi & \textbf{lent/tard} \\%8
				\hline
				つまらない & tsumaranai & \textbf{ennuyeux} \\%9
				\hline
				まずい & mazui & \textbf{fade} \\%10
				\hline
				わるい & warui & \textbf{mauvais} \\%11
				\hline
				かるい & karui & \textbf{léger} \\%12
				\hline
				かなしい & kanashii & \textbf{triste} \\%13
				\hline
				くらい & kurai & \textbf{sombre} \\%14
				\hline
%				 &  & \textbf{} \\%15
%				\hline
%				 &  & \textbf{} \\%16
%				\hline
%				 &  & \textbf{} \\%17
%				\hline
%				 &  & \textbf{} \\%18
%				\hline
%				 &  & \textbf{} \\%19
%				\hline
%				 &  & \textbf{} \\%20
%				\hline
			\end{tabular}
		\end{multicols}
		~\\
		\begin{tabular}{|c|c|c|}
			\hline
			\textbf{Hiragana} & \textbf{Rõmaji} & \textbf{Français} \\
			\hline
			\textbf{occupé} & isogashii & いそがし \\
			\hline
			\textbf{rouge} & akai & \textcolor{red}{あかい} \\
			\hline
			\textbf{bleu} & aoi & \textcolor{blue}{あおい} \\
			\hline
			\textbf{blanc} & shiroi & \textcolor[rgb]{0.75,0.75,0.75}{しろい} \\
			\hline
			\textbf{noir} & kuroi & \textcolor{black}{くろい} \\
			\hline
			\textbf{jaune} & kiiroi & \textcolor[rgb]{0.9,0.8,0}{きいろい} \\
			\hline
%			\textbf{} &  &  \\
%			\hline
%			\textbf{} &  &  \\
%			\hline
%			\textbf{} &  &  \\
%			\hline
%			\textbf{} &  &  \\
%			\hline
%			\textbf{} &  &  \\
%			\hline
%			\textbf{} &  &  \\
%			\hline
%			\textbf{} &  &  \\
%			\hline
%			\textbf{} &  &  \\
%			\hline
		\end{tabular}
	\end{center}
\end{Japanese}  
\end{CJK}

\chapter{Adjectifs en na}

\begin{CJK}{UTF8}{}  
\begin{Japanese}
	\begin{center}
		\begin{multicols}{2}
			\begin{tabular}{|c|c|c|}
				\hline
				\textbf{Français} & \textbf{Rõmaji} & \textbf{Hiragana} \\
				\hline
				\textbf{pratique} & benri & べんり \\%0
				\hline
				\textbf{calme} & shizuka & しずか \\%1
				\hline
				\textbf{préféré} & suki & すき \\%2
				\hline
				\textbf{doué} & jouzu & じょうず \\%3
				\hline
%				\textbf{grand} & ookii & おおきい \\%3
%				\hline
%				\textbf{long} & nagai & ながい \\%5
%				\hline
%				\textbf{difficile} & muzukashii & むずかしい \\%6
%				\hline
%				\textbf{neuf} & atarashii & あたらしい \\%7
%				\hline
%				\textbf{rapide/t\^ot} & hayai & はやい \\%8 %ma touche pour l'accent circonflexe veux pas fonctinner : (
%				\hline
%				\textbf{interessant} & omoshiroi & おもしろい \\%9
%				\hline
%				\textbf{délicieux} & oishii & おいしい \\%10
%				\hline
%				\textbf{bon} & ii & いい \\%11
%				\hline
%				\textbf{lourd} & omoi & おもい \\%12
%				\hline
%				\textbf{} &  &  \\%13
%				\hline
%				\textbf{} &  &  \\%14
%				\hline
%				\textbf{} &  &  \\%15
%				\hline
%				\textbf{} &  &  \\%16
%				\hline
%				\textbf{} &  &  \\%17
%				\hline
%				\textbf{} &  &  \\%18
%				\hline
%				\textbf{} &  &  \\%19
%				\hline
%				\textbf{} &  &  \\%20
%				\hline

			\end{tabular}
			\begin{tabular}{|c|c|c|}
				\hline
				\textbf{Hiragana} & \textbf{Rõmaji} & \textbf{Français} \\
				\hline
				ふべん & huben & \textbf{pas pratique} \\%0
				\hline
				にぎやか & nigiyaka & \textbf{animé} \\%1
				\hline
				きらい & kirai & \textbf{qu'on aime pas} \\%2
				\hline
				へた & heta & \textbf{pas doué} \\%3
				\hline
%				ちいさい & chiisai & \textbf{petit} \\%4
%				\hline
%				みじかい & mijikai & \textbf{court} \\%5
%				\hline
%				やさしい & yasashii & \textbf{facile} \\%6
%				\hline
%				ふるい & furui & \textbf{ancien} \\%7
%				\hline
%				おそい & osoi & \textbf{lent/tard} \\%8
%				\hline
%				つまらない & tsumaranai & \textbf{ennuyeux} \\%9
%				\hline
%				まずい & mazui & \textbf{fade} \\%10
%				\hline
%				わるい & warui & \textbf{mauvais} \\%11
%				\hline
%				かるい & karui & \textbf{léger} \\%12
%				\hline
%				 &  & \textbf{} \\%13
%				\hline
%				 &  & \textbf{} \\%14
%				\hline
%				 &  & \textbf{} \\%15
%				\hline
%				 &  & \textbf{} \\%16
%				\hline
%				 &  & \textbf{} \\%17
%				\hline
%				 &  & \textbf{} \\%18
%				\hline
%				 &  & \textbf{} \\%19
%				\hline
%				 &  & \textbf{} \\%20
%				\hline
			\end{tabular}
		\end{multicols}
	\end{center}
\end{Japanese}  
\end{CJK}

\end{document}