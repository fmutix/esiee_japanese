\documentclass[11pt]{report}
\usepackage[top=2cm, bottom=2cm, left=2cm, right=2cm]{geometry}
\usepackage[T1]{fontenc}
\usepackage[utf8]{inputenc}
\usepackage[french]{babel}
\usepackage{lmodern}
\usepackage{hyperref}
\usepackage{titlesec, blindtext}
\usepackage{multirow}
\usepackage{listings}
\usepackage{color}
\usepackage{fixltx2e}
\usepackage{graphicx}
\usepackage{amsmath}
\usepackage{CJKutf8}
\newenvironment{Japanese}{%  
\CJKfamily{min}%  
\CJKtilde  
\CJKnospace}{} 

\definecolor{dkgreen}{rgb}{0,0.6,0}
\definecolor{gray}{rgb}{0.5,0.5,0.5}
\definecolor{mauve}{rgb}{0.58,0,0.82}
\definecolor{blue}{rgb}{0,0,0.7}
\renewcommand{\arraystretch}{1.2}
\newcommand{\hsp}{\hspace{20pt}}
\titleformat{\chapter}[hang]{\Huge\bfseries}{\thechapter\hsp{|}\hsp}{0pt}{\Huge\bfseries}
\hypersetup{
	colorlinks=true,       	% false: boxed links; true: colored links
	linkcolor=black,          	% color of internal links
	urlcolor=blue           	% color of external links
}

\title{Cours de japonais}
\author{
	Frédéric Nguyen \\ Club*Nix
}

\begin{document}
\maketitle
\tableofcontents

\chapter{Adjectifs forme en i}

\section{Formes}

\begin{CJK}{UTF8}{}  
\begin{Japanese}
	\begin{center}
		\begin{tabular}{|c|c|c|}
			\hline
			\textbf{Forme} & \textbf{Roumaji} & \textbf{Hiragana} \\
			\hline
			Affirmative & atsui & あつい \\
			\hline
			Négative & atsukunai & あつくない \\
			\hline
			Affirmative passée & atsukatta & あつかった \\
			\hline
			Négative passée & tsukunakatta & あtしゅくなかった \\
			\hline 
		\end{tabular}
	\end{center}
	Il y a un irrégulier qui est l'adjectif bon [いい]. \\
	- いい \\
	- よくない \\
	- よかった \\
	- よくなかった \\ \\
	
	Autre forme d'adjectif dans les phrases pour dire "ça a l'air ..." \\
	Les hiraganas sont faciles : ひらがなは やさしい です。 \\
	Les hiraganas ont l'air facile : ひらがなは やさしそう です。 \\
\end{Japanese}  
\end{CJK}  

\chapter{Verbes}

\section{Forme en te}

\begin{CJK}{UTF8}{}  
\begin{Japanese}
	\begin{description}
		\item[Ichidan] \hfill \\
			たべます -> たべ\textcolor{red}{て}
		\item[Godan] \hfill \\
			の\textcolor{red}{み}ます -> の\textcolor{red}{んで} \\
			と\textcolor{red}{り}ます -> と\textcolor{red}{って} \\
			か\textcolor{red}{い}ます -> か\textcolor{red}{って} \\
			ま\textcolor{red}{ち}ます -> ま\textcolor{red}{って} \\
			か\textcolor{red}{き}ます -> か\textcolor{red}{いて} \\
			ぬ\textcolor{red}{ぎ}ます -> ぬ\textcolor{red}{いで}
		\item[Irrégulier] \hfill \\
			します -> して \\
			いきます -> いって
	\end{description}
	\hfill \\
	Utilisation possibles avec la forme en te : \\ \\
	1. のんで ください。 \\
	Buvez-s'il vous plait. \\ \\
	2. わtしは サンドイッチを たべて います。 \\
	Je suis en train de manger un sandwich. \\ \\
	3. かいても いいですか。 \\
	Puis-je écrire ?
\end{Japanese}  
\end{CJK}

\section{Forme négative}

\begin{CJK}{UTF8}{}  
\begin{Japanese}
	\begin{description}
		\item[Ichidan] \hfill \\
			たべます -> たべない
		\item[Godan] \hfill \\
			の\textcolor{red}{み}ます -> の\textcolor{red}{ま}ない \\
			と\textcolor{red}{り}ます -> と\textcolor{red}{ら}ない \\
			か\textcolor{red}{い}ます -> か\textcolor{red}{わ}ない \\
			ま\textcolor{red}{ち}ます -> ま\textcolor{red}{さ}ない \\
			はな\textcolor{red}{し}ます -> はな\textcolor{red}{さ}ない \\
			か\textcolor{red}{き}ます -> か\textcolor{red}{か}ない \\
			ぬ\textcolor{red}{ぎ}ます -> ぬ\textcolor{red}{が}ない
		\item[Irrégulier] \hfill \\
			します -> しない \\
			きます -> こない
	\end{description}
	\hfill \\
	Exemples : \\ \\
	1. この いぬを たべないで ください。 \\
	Ne mange pas cette viande. \\ \\
	2. まどを あけないで ください。\\
	N'ouvre pas les fenêtres.
\end{Japanese}  
\end{CJK}

\section{Forme neutre}

\begin{CJK}{UTF8}{}  
\begin{Japanese}
	\begin{description}
		\item[Ichidan] \hfill \\
			たべます -> たべる
		\item[Godan] \hfill \\
			の\textcolor{red}{み}ます -> の\textcolor{red}{む} \\
			と\textcolor{red}{り}ます -> と\textcolor{red}{る} \\
			か\textcolor{red}{い}ます -> か\textcolor{red}{わ} \\
			ま\textcolor{red}{ち}ます -> ま\textcolor{red}{つ} \\
			はな\textcolor{red}{し}ます -> はな\textcolor{red}{す} \\
			か\textcolor{red}{き}ます -> か\textcolor{red}{く} \\
			ぬ\textcolor{red}{ぎ}ます -> ぬ\textcolor{red}{ぐ}
		\item[Irrégulier] \hfill \\
			します -> する \\
			きます -> くる
	\end{description}
\end{Japanese}  
\end{CJK}

\chapter{Obligation}

\section{Forme d'obligation}

\begin{CJK}{UTF8}{}  
\begin{Japanese}
	\underline{\emph{verbe}}なければなりません。-> Il faut... \\
	
	Exemples : \\ \\
	1. まいあさ ごはんを \underline{たべ}なければなりません。 (\underline{たべ}ます)\\
	Il faut manger tous les jours. \\ \\
	2. くすりを のみますから \underline{ね}なければなりません。(\underline{ね}ます) \\
	Il faut dormir après avoir bu ses médicaments. \\ \\
	3. この まんがを \underline{かい}なければなりません。 (\underline{かい}ます)\\
	Il faut acheter ce manga.
\end{Japanese}  
\end{CJK}

\section{Non obligation}

\begin{CJK}{UTF8}{}  
\begin{Japanese}
	\underline{\emph{verbe}}なくてもいいです。-> Ce n'est pas la peine, pas obligé de... \\
	
	Exemples : \\ \\
	1. この てがみを \underline{かき}なくてもいいです。 (\underline{かき}ます) \\
	C'est n'est pas la peine d'écrire cette lettre. \\ \\
	2. たこさんを \underline{よび}なくてもいいです。 (\underline{よび}ます) \\
	Ce n'est pas la peine d'appeler monsieur Poulpe. \\ \\
	3. あさっては がっこうへ いきなくてもいいです。 \\
	Ce n'est pas la peine d'aller à \, l'école après-demain.
\end{Japanese}  
\end{CJK}

\end{document}