\documentclass[11pt]{report}
\usepackage[top=2cm, bottom=2cm, left=2cm, right=2cm]{geometry}
\usepackage[T1]{fontenc}
\usepackage[utf8]{inputenc}
\usepackage[french]{babel}
\usepackage{lmodern}
\usepackage{hyperref}
\usepackage{titlesec, blindtext}
\usepackage{multirow}
\usepackage{listings}
\usepackage{color}
\usepackage{fixltx2e}
\usepackage{graphicx}
\usepackage{amsmath}
\usepackage{CJKutf8}
\newenvironment{Japanese}{%  
\CJKfamily{min}%  
\CJKtilde  
\CJKnospace}{} 

\definecolor{dkgreen}{rgb}{0,0.6,0}
\definecolor{gray}{rgb}{0.5,0.5,0.5}
\definecolor{mauve}{rgb}{0.58,0,0.82}
\definecolor{blue}{RGB}{93,120,237}
\renewcommand{\arraystretch}{1.2}
\newcommand{\hsp}{\hspace{20pt}}
\titleformat{\chapter}[hang]{\Huge\bfseries}{\thechapter\hsp{|}\hsp}{0pt}{\Huge\bfseries}
\hypersetup{
	colorlinks=true,       	% false: boxed links; true: colored links
	linkcolor=black,          	% color of internal links
	urlcolor=blue           	% color of external links
}

\title{Japonais - Grammaire}
\author{
	Club*Nix
}

%\begin{CJK}{UTF8}{}  
%\begin{Japanese}
%\end{Japanese}  
%\end{CJK}

\begin{document}
\maketitle
\tableofcontents

\chapter{Adjectifs}

\section{Adjectifs en i}

\begin{CJK}{UTF8}{}  
\begin{Japanese}
	\begin{flushleft}
	Un adjectif est plac\'e juste devant le nom qu'il qualifie : \\
	わたしは \textcolor{blue}{おもしろい} ほんを よみました。 \\
	J'ai lu un livre intéressant.
	\end{flushleft}
\end{Japanese}  
\end{CJK}

\subsection{Les formes}

\begin{CJK}{UTF8}{}  
\begin{Japanese}
Les adjectif en i peuvent prendre plusieurs formes pour donner différents sens \`a la phrase.
	\begin{center}
		\begin{tabular}{|c|c|c|}
			\hline
			\textbf{Forme} & \textbf{Roumaji} & \textbf{Hiragana} \\
			\hline
			Affirmative & atsui & あつ\textcolor{red}{い} \\
			\hline
			Négative & atsukunai & あつ\textcolor{red}{くない} \\
			\hline
			Affirmative passée & atsukatta & あつ\textcolor{red}{かった} \\
			\hline
			Négative passée & atsukunakatta & あつ\textcolor{red}{くなかった} \\
			\hline 
		\end{tabular}
	\end{center}
	Il y a un irrégulier qui est l'adjectif bon [いい]. \\
	- いい \\
	- よくない \\
	- よかった \\
	- よくなかった
\end{Japanese}  
\end{CJK}

\subsection{Addition d'adjectifs}

\begin{CJK}{UTF8}{}  
\begin{Japanese}
	On peut attribuer plusieurs adjectifs à un même nom en mettant tous les adjectifs à la suite, mais il faut changer le caractère \textcolor{red}{い} \, en \textcolor{red}{くて}, mais le dernier adjectif devra rester sans modification. \\ \\
	\textcolor{blue}{あたらし\underline{くて}} \textcolor{blue}{たか\underline{い}} くるま。 \\
	Une voiture nouvelle et chère.
\end{Japanese}  
\end{CJK}

\subsection{Impression}

%TODO REDO

\begin{CJK}{UTF8}{}  
\begin{Japanese}
	\begin{flushleft}
	Autre forme d'adjectif dans les phrases pour dire "ça a l'air ..." \\
	Les hiraganas sont faciles : ひらがなは \textcolor{blue}{やさしい} です。 \\
	Les hiraganas ont l'air facile : ひらがなは \textcolor{blue}{やさし\underline{そう}} です。
	\end{flushleft}
\end{Japanese}  
\end{CJK}

\subsection{Avec devenir}

\begin{CJK}{UTF8}{}  
\begin{Japanese}
	Construction de base : \underline{radical adj-i}く なります \\ \\
	1. この まんが は だんだん \underline{おもしろ\textcolor{red}{く}} なります。 \\
	Ce livre devient de plus en plus intéressant. \\ \\
	2. こども は \underline{おおき\textcolor{red}{く}} なります。 \\
	Les enfants deviennent grands.
\end{Japanese}  
\end{CJK}

\section{Adjectifs en na}

\begin{CJK}{UTF8}{}  
\begin{Japanese}
	Avec les adjectifs en na, on peut le placer dans la phrase comme un adjectif en i, mais il faut ajouter la particule \textcolor{red}{な} \,juste après l'adjectif. \\ \\
	\textcolor{blue}{きれい\underline{な}} はな。\\
	Une jolie fleur.
\end{Japanese}  
\end{CJK}

\subsection{Négation}

\begin{CJK}{UTF8}{}  
\begin{Japanese}
	
\end{Japanese}  
\end{CJK}

\subsection{Addition d'adjectifs}

\begin{CJK}{UTF8}{}  
\begin{Japanese}
	Aussi comme les adjectifs en i, on peut lister les adjectifs \`a la suite en remplaçant le \textcolor{red}{な} \,par \textcolor{red}{で} \,et en laissant le dernier adjectif intact. \\ \\
	1. \textcolor{blue}{きれい\underline{で}} ちいさい はな。 \\
	Une belle et petite fleur. \\ \\
	2. やすくて にぎやかな レストラン。 \\
	Un restaurant anim\'e et pas cher.
\end{Japanese}  
\end{CJK}

\subsection{Avec devenir}

\begin{CJK}{UTF8}{}  
\begin{Japanese}
	Construction de base : \underline{adj-na} に なります \\ \\
	3. パソコン ハ ダンダン \underline{べんり\textcolor{red}{に}} なります。 \\
	Les ordinateurs deviennent de plus en plus pratiques. \\ \\
	4. わたし は テニス が \underline{じょうず\textcolor{red}{に}} なります。 \\
	Je deviens meilleur au tennis.
\end{Japanese}  
\end{CJK}

\chapter{Verbes}

\section{Forme en masu}

\begin{CJK}{UTF8}{}  
\begin{Japanese}
	La forme avec laquelle on a commenc\'e \`a apprendre les verbes. Est aussi marque de politesse. \\
	\begin{description}
		\item[Présent] \hfill \\
		たべます
		\item[Négatif] \hfill \\
		たべません
		\item[Passé] \hfill \\
		たべました
		\item[Négatif passé] \hfill \\
		たべません でした
	\end{description}
\end{Japanese}  
\end{CJK}

\section{Forme en te}

\begin{CJK}{UTF8}{}  
\begin{Japanese}
	\begin{description}
		\item[Ichidan] \hfill \\
			たべます -> たべ\textcolor{red}{て}
		\item[Godan] \hfill \\
			の\textcolor{red}{み}ます -> の\textcolor{red}{んで} \\
			と\textcolor{red}{り}ます -> と\textcolor{red}{って} \\
			か\textcolor{red}{い}ます -> か\textcolor{red}{って} \\
			ま\textcolor{red}{ち}ます -> ま\textcolor{red}{って} \\
			か\textcolor{red}{き}ます -> か\textcolor{red}{いて} \\
			ぬ\textcolor{red}{ぎ}ます -> ぬ\textcolor{red}{いで}
		\item[Irrégulier] \hfill \\
			します -> して \\
			いきます -> いって
	\end{description}
\end{Japanese}  
\end{CJK}

\subsection{Utilisations}

\begin{CJK}{UTF8}{}  
\begin{Japanese}
	\begin{description}
		\item[Le S'il vous plaît : \textcolor{blue}{\underline{v}て ください}] \hfill \\
			1. のんで ください。 \\
			Buvez-s'il vous plait.
		\item[Action en cours : \textcolor{blue}{\underline{v}て います}] \hfill \\
			2. わtしは サンドイッチを たべて います。 \\
			Je suis en train de manger un sandwich.
		\item[Permission : \textcolor{blue}{\underline{v}ても いいですか}] \hfill \\
			3. かいても いいですか。 \\
			Puis-je écrire ?
		\item[Addition de verbes : \textcolor{blue}{\underline{v}て \underline{v}て \underline{v}ます}] \hfill \\
			4. わたしは すしを たべて テレビを みます。 \\
			Je mange des sushis et je regarde la télévision.  \\
			\item[Etat en cours] \hfill \\
			\begin{description}
				\item[Habiter : \textcolor{blue}{\underline{\qquad} に すんで います}] \hfill \\
				5. わたし は フランス に すんで います。 \\
				J'habite en France.
				\item[Connaître] \hfill \\
				6. わたし は ピカチュ を しって います。 \\
				Je connais Pikachu. \\
				\emph{negatif : しりません}
				\item[Possession : \textcolor{blue}{\underline{\qquad}を もって います}] \hfill \\
				7. わたし は いぬ をもって います。 \\
				Je possède un chien.
				\item[Etre mari\'e] \hfill \\
				8. けっこん して います。 \\
				Je suis marié. \\
				\emph{Célibataire : どくしん}
			\end{description}
	\end{description}
\end{Japanese}  
\end{CJK}

\section{Forme en a}

\begin{CJK}{UTF8}{}  
\begin{Japanese}
	\begin{description}
		\item[Ichidan] \hfill \\
			たべます -> たべない
		\item[Godan] \hfill \\
			の\textcolor{red}{み}ます -> の\textcolor{red}{ま}ない \\
			と\textcolor{red}{り}ます -> と\textcolor{red}{ら}ない \\
			か\textcolor{red}{い}ます -> か\textcolor{red}{わ}ない \\
			ま\textcolor{red}{ち}ます -> ま\textcolor{red}{さ}ない \\
			はな\textcolor{red}{し}ます -> はな\textcolor{red}{さ}ない \\
			か\textcolor{red}{き}ます -> か\textcolor{red}{か}ない \\
			ぬ\textcolor{red}{ぎ}ます -> ぬ\textcolor{red}{が}ない
		\item[Irrégulier] \hfill \\
			します -> しない \\
			きます -> こない
	\end{description}
\end{Japanese}  
\end{CJK}

\subsection{Utilisations}

\begin{CJK}{UTF8}{}  
\begin{Japanese}
	\begin{description}
		\item[Négation : \textcolor{blue}{\underline{v}ないで ください}] \hfill \\
		1. この ほんを のまないで ください。 \\
		Ne lit pas ce livre.
		\item[Obligation : \textcolor{blue}{\underline{$v_{a}$}なければ なりません}] \hfill \\
		2. くすりを のみますから \underline{ね}なければなりません。 \\
		Il faut dormir après avoir bu ses médicaments.
		\item[Non obligation : \textcolor{blue}{\underline{$v_{a}$}なくても いいです}] \hfill \\
		3. あさっては がっこうへ いきなくてもいいです。 \\
		Ce n'est pas la peine d'aller \`a l'école après-demain.
	\end{description}
\end{Japanese}
\end{CJK}

\section{Forme neutre}

\begin{CJK}{UTF8}{}  
\begin{Japanese}
	\begin{description}
		\item[Ichidan] \hfill \\
			たべます -> たべる
		\item[Godan] \hfill \\
			の\textcolor{red}{み}ます -> の\textcolor{red}{む} \\
			と\textcolor{red}{り}ます -> と\textcolor{red}{る} \\
			か\textcolor{red}{い}ます -> か\textcolor{red}{わ} \\
			ま\textcolor{red}{ち}ます -> ま\textcolor{red}{つ} \\
			はな\textcolor{red}{し}ます -> はな\textcolor{red}{す} \\
			か\textcolor{red}{き}ます -> か\textcolor{red}{く} \\
			ぬ\textcolor{red}{ぎ}ます -> ぬ\textcolor{red}{ぐ}
		\item[Irrégulier] \hfill \\
			します -> する \\
			きます -> くる
	\end{description}
\end{Japanese}  
\end{CJK}

\subsection{Utilisations}

\begin{CJK}{UTF8}{}  
\begin{Japanese}
	\begin{description}
		\item[Savoir faire : \textcolor{blue}{\underline{vfn} こと が できます}] \hfill \\
		1. わたし は ひらがな かく こと が できます。 \\
		Je sais écrire les hiraganas. \\ \\
		2. うまに のる こと が です。 \\
		Je sais monter \`a cheval. \\
		\emph{Attention, la fin de phrase change quand on fait avec un verbe. Ici, uma ni norimasu c'est le verbe monter à cheval.}
		\item[Avant de : \textcolor{blue}{\underline{vfn} まえに \underline{\qquad}ます}] \hfill \\
		3. わたし は \textcolor{red}{しごとに いく} まえに \textcolor{blue}{くすりを のみます}。 \\
		Je \textcolor{blue}{bois mes médicaments} avant d'\textcolor{red}{aller au travail}.
		\item[Hobbies : \textcolor{blue}{しゅみ は \underline{nom/qqchを vfn} です}] \hfill \\
		4. しゅみ は ピアノを ひく です。 \\
		Mon hobby est de jouer au piano.
	\end{description}
\end{Japanese}  
\end{CJK}

\chapter{Comparaison}

\section{Comparatif}

\subsection{Comparaison simple}

\begin{CJK}{UTF8}{}  
\begin{Japanese}
	Pour comparer un objet A \`a un objet B, on utilise \textcolor{red}{より}. \\ \\
	1. \textcolor{red}{ねこ} は \textcolor{blue}{いぬ} \underline{より} かっくい です。 \\
	Comparés aux \textcolor{blue}{chiens}, les \textcolor{red}{chats} sont plus cools. \\ \\
	2. \textcolor{red}{パソコン} は \textcolor{blue}{車} より やすい です。 \\
	Comparés aux \textcolor{blue}{voitures}, les \textcolor{red}{ordinateurs} sont \`a plus petit prix.
\end{Japanese}  
\end{CJK}

\subsection{Question}

\begin{CJK}{UTF8}{}  
\begin{Japanese}
	Pour demander une comparaison entre deux choses, on liste les deux objets avec la particule と \, après chacun puis on fait suivre avec \textcolor{red}{どちらが} \, pour préciser lequel d'entre eux est le plus qualifi\'e par l'adjectif. (Explication dont on peut se passer...). \\ \\
	3. にほんごと えいごと どちら が たさしい ですか。 \\
	Quel est le plus facile entre le japonais et l'anglais ? \\ 
	\\
	Mais la réponse à cette question est spécifique dans sa forme. En comparant A avec B, il faut imaginer que A et B forment deux camps et l'adjectif se trouve entre les deux. Il faut dire dans quel camp serait l'adjectif (c'est ma façon de voir). \\
	\\
	4. えいごの ほう が やさしい です。 \\
	L'anglais est plus facile.
\end{Japanese}  
\end{CJK}

\section{Superlatif}

\begin{CJK}{UTF8}{}  
\begin{Japanese}
	Le superlatif permet de désigner un numéro1 \`a partir d'une liste d'éléments (minimum 3) ou d'un thème en général. Et une petite forme générale : \\
	Thème/liste, interrogatif(doko, dare...) が いちばん \, adj ですか。 \\ \\
	5. \textcolor{blue}{あなたの ともだち}で だれ が いちばん \underline{せがたかい} ですか。 \\
	Parmi \textcolor{blue}{tes amis}, qui est le plus \underline{grand} ? \\ \\
	6. くるまと ひこうきと じでんしゃと なに が いちばん はやい ですか。 \\
	Entre les voitures, les avions et les vélos, qu'est-ce qui est le plus rapide ?
	
\end{Japanese}  
\end{CJK}

\end{document}